\subsection{Step 3 : Assuming polyhedron particle}

For now we studied 4 shapes : regular tetrahedron, cube, regular pyramid and regular triangular prism :

\includegraphics[width=0.8\textwidth]{Images/p_shapes.png}

In each case, we will consider all edges of the same length.

\subsubsection{CLD plots}

Here, we use Monte-Carlo to obtain the $CLD(l)$ plots for each shape.

Method : 
\begin{itemize}
    \item Generate a basic tetrahedron (or cube, pyramid or prism), with known position and orientation.
    \item Rotate it uniformly randomly.
    \item Project it on the x-y plane.
    \item generate a chord along the y-axis with a uniform random y-coordinate.
    \item Find the intersection of the chord with the projection. Its length is the chord length.
    \item Save the chord length.
    \item Repeat the process from step 1.
    \item Plot the CLDs as histograms.
\end{itemize}

\vspace{0.5cm} 

Note : There is two types of CLD plots : 
\begin{itemize}
    \item The cumulative one : $CLD(l) = Q(l) = P(L \leq l)$
    \item The differential one : $CLD(l) = q(l) = P(L = l) (density)$
\end{itemize}

The obtained plots are as follows :